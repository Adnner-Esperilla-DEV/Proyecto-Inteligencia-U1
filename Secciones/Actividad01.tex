%%%%%%%%%%%%%%%%%%%%%%%%%%%%%%%%%%%%%%%%%%%%%%%%%%%%%%%%%%%%%%%%%%%%%%%%%%%
%
% Plantilla para un artículo en LaTeX en español.
%
%%%%%%%%%%%%%%%%%%%%%%%%%%%%%%%%%%%%%%%%%%%%%%%%%%%%%%%%%%%%%%%%%%%%%%%%%%%



%--------------------------------------------------------------------------
\title{Plantilla para un artículo \LaTeX}
\author{El autor va aquí\\
  \small Dept. Plantillas y Editores\\
  \small E12345\\
  \small España
}

\begin{document}
\section{Resumen}
\item{Un sistema de recomendaci´on es un sistema inteligente que proporciona a los usuarios una serie de sugerencias personalizadas (recomendaciones) sobre un determinado tipo de elementos (´ıtems). Los sistemas de recomendaci´on estudian las caracter´ısticas de cada usuario y mediante un procesamiento de los datos, encuentra un subconjunto de ´ıtems que pueden resultar de inter´es para el usuario. En los ´ultimos a˜nos y debido principalmente a la sobre carga de informaci´on que tenemos en internet, han proliferado los sistemas de recomendaci´on, los cuales proporcionan a los usuarios, informaci´on, productos, etc. que puedan ser del inter´es del usuario, tras realizar un . estudio”de su perfil, su gusto e incluso de la forma en la que el usuario navega por internet.

\section{Abstract}
\item{A recommendation system is an intelligent system that provides users with a series of personalized suggestions (recommendations) about a certain type of elements (items). The recommendation systems study the characteristics of each user and through a processing of the data, find a subset of items that may be of interest to the user. In recent years and mainly due to the overload of information that we have on the Internet, recommendation systems have proliferated, which provide users, information, products, etc. that may be of interest to the user, after conducting a ”study .of your profile, your taste and even the way in which the user browses the internet.
\newpage

\section{Introduccion}
\item{Los sistemas de recomendaci´on pueden definirse como herramientas dise˜nadas para interactuar con conjuntos de informacion grandes y complejos con la finalidad de proporcionar al usuario informacion o ıtems que sean de su inter´es, todo ello de forma automatizada. Su funcionamiento se basa en el empleo de m´etodos matem´aticos y estad´ısticos capaces de explotar la informaci´on previamente almacenada y crear recomendaciones adaptadas a cada usuario. En la actualidad, los sistemas de recomendaci´on son una tecnolog´ıa implementada en la mayor´ıa de plataformas online como Amazon, Neflix, eBay. . . ya que han dado muy buenos resultados incrementando las ventas. Tambi´en est´an presentes en muchos otros ´ambitos, por ejemplo, el de las noticias, mostrando al usuario informaci´on que le interesa de forma r´apida. La mayor´ıa de sistemas de recomendaci´on se pueden clasificar en tres grupos: basados en contenido, filtrado colaborativos y mixtos (combinaci´on de los dos anteriores). El objetivo de los ejemplos mostrados en este documento es facilitar la comprensi´on de las ideas que hay detr´as de algunos de estos sistemas, no persiguen ser una implementaci´on ´optima y sofisticada, sino intuitiva. Para sistemas m´as optimizados pueden emplearse librer´ıas como recommenderlab.}

\section{   TITULO}
\item{  Sistema de recomendacion de frameworks de desarrollo de aplicaciones}
\begin{figure}[htb]
\begin{center}
\includegraphics[width=15cm]{./Imagenes/imagen1}
\end{center}
\end{figure}

\item{.
}
\section{AUTORES}

\item{-- Jhony Mamani Limache\\ -- Nilson Laura Atencio\\ -- Adnner Esperilla Ruiz\\ -- Lisbeth Espinoza \\ -- Marlon Villegas Arando
}
\newpage
\section{PLANTEAMIENTO DEL PROBLEMA}
\subsection{Problema}

\item{Muchos desarrolladores al momento de empezar a desarrollar aplicaciones en una plataforma, no sabe que framework escoger y cual sea m´as practico para el desarrollo de su aplicacion.}

\subsection{Justificacion}
\item{falta de orientacion al escoger un framework de desarrollo.}

\subsection{Alcance}
\item{ a todos los desarrolladores que piensan implementar un sistema.}

\section{OBJETIVOS}

\subsection{General}
\item{  orientar al usuario a escoger un framework de desarrollo segun sus objetivos y aprendizaje de lenguajes de programacion anteriores.}

\subsection{Especificos}
\item{,....}

\section{REFERENTES TEORICOS}
\item{,....}


\newpage

\section{DESARROLLO DE LA PROPUESTA}

\subsection{Tecnologia de informacion: }
\item{....}
\subsection{Metodologia , Tecnicas usadas: }
\item{,....}

\section{CRONOGRAMA (PERSONAS, TIEMPO,OTROS RECURSOS)}

\item{.... }

\begin{figure}[htb]
\begin{center}
\includegraphics[width=15cm]{./Imagenes/cronograma_01.jgp}
\end{center}
\end{figure}


\newpage

\section{Conclusiones}

\begin{itemize}
\item 

\item 


\item 

\item 
\end{itemize}


\newpage
% Bibliografía.
%-----------------------------------------------------------------
\begin{thebibliography}{99}


\bibitem{Cd94} referencia 1
\bibitem{Cd94}  referencia 2
\bibitem{Cd94}  referencia 3

\end{thebibliography}

\end{document}
