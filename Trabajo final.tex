%
\documentclass[%
 reprint,
 amsmath,amssymb,
 aps,
]{revtex4-1}

\usepackage{graphicx}% Include figure files
\usepackage{dcolumn}% Align table columns on decimal point
\usepackage{bm}% bold math


\begin{document}



\title{Trabajo Final U I  \\ BOT Recomendacion de Framework para desarrollo de aplicaciones }
\author{Jhony Mamani Limache}
\author{Nilson Laura Atencio}
\author{Adnner Esperilla Ruiz}
\author{Lisbeth Espinoza}
\author{Marlon Villegas Arando}
\affiliation{%
 Universidad Privada de Tacna \textbackslash Facultad de Ingenieria \textbackslash Escuela Profesional de Ingenieria de Sistemas
}%

\begin{abstract}
\begin{center}
\textbf{Resumen}
\end{center}

En este artículo aprenderemos sobre la seguridad de los datos y lo que implica protegerlos de operaciones indebidas que pongan en peligro su definición, existencia, consistencia e integridad independientemente de la persona que los accede. Algunos comandos que se emplean para el objetivo del articulo es GRANT, REVOKE\\

\textbf{Palabras clave:}   seguridad, integridad, GRANT, REVOKE.\\

\begin{center}
\textbf{Abstract}
\end{center}
In this article we will learn about data security and what it means to protect data from improper operations that jeopardize its definition, existence, consistency and integrity regardless of who accesses it. Some commands used for the purpose of the article is GRANT, REVOKE\\
\textbf{Keywords:}   security, integrity, GRANT, REVOKE.\\

\end{abstract}



\maketitle

%\tableofcontents

\section {Introducción}\label{sec:1}

En el mundo de hoy, cada usuario o cliente de las diferentes plataformas online se enfrenta a múltiples opciones. Por ejemplo: si estoy buscando una película para ver sin tener una idea específica de lo que quiero, hay una amplia gama de posibilidades en cómo mi búsqueda podría tener éxito. Es posible que pierda mucho tiempo navegando por Internet y repasando varios sitios con la esperanza de encontrar recomendaciones de diferentes personas con diferentes gustos

%-----------------------------------------------------------------


\section {Titulo}\label{sec:1}

    recomendacion de frameworks de desarrollo de aplicaciones
%-----------------------------------------------------------------
\section{Autores}\label{sec:2}


%-----------------------------------------------------------------
\section {Planteamiento del problema}

\subsection{Problema}
Muchos desarrolladores al momento de empezar a desarrollar aplicaciones en una plataforma, no sabe que framework escoger y cual sea más practico para el desarrollo de su aplicacion.

\subsection{Justificacion}
falta de orientacion al escoger un framework de desarrollo
\subsection{Alcance}
a todos los desarrolladores que piensan implementar un sistema.

%-----------------------------------------------------------------
\section{Objetivos}\label{sec:2}
\subsection{General:}
-  orientar al usuario a escoger un framework de desarrollo segun sus objetivos y aprendizaje de lenguajes de programacion anteriores
\subsection{Específicos:}
-  Definir conceptos sobre seguridad de base de datos, privilegios y autorizaciones.\\
- Comparar las definiciones.

%-----------------------------------------------------------------
\section {Referentes Teoricos}

%-----------------------------------------------------------------
\section {Desarrollo de la propuesta}

\subsection{Tecnología de información:}
-  Determinar como es que se establece la seguridad en las bases de datos.
\subsection{Metodología, técnicas usadas:}



%-----------------------------------------------------------------
\section {Cronograma (personas, tiempo, otros recursos)}

ffff


% Bibliografia.
%-----------------------------------------------------------------

\bibliographystyle{plain}
\bibliography{Bibliografia}

\end{document}
